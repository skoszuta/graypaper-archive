\section{Grandpa and the Best Chain}\label{sec:bestchain}\label{sec:grandpa}

Nodes take part in the \textsc{Grandpa} protocol as defined by \cite{stewart2020grandpa}.

\newcommand{\final}{\natural}
\newcommand{\best}{\flat}

We define the latest finalized block as $\mathbf{B}^\final$. All associated terms concerning block and state are similarly superscripted. We consider the \emph{best block}, $\mathbf{B}^\best$ to be that which is drawn from the set of acceptable blocks of the following criteria:

\begin{itemize}
  \item Has the finalized block as an ancestor.
  \item Contains no unfinalized blocks where we see an equivocation (two valid blocks at the same timeslot).
  \item Is considered audited.
\end{itemize}

Formally:
\begin{align}
  \mathbf{A}(\mathbf{H}^\best) &\owns \mathbf{H}^\final \\
  \mathbf{U}^\best &\equiv \top \\
  \not\exists \mathbf{H}^A, \mathbf{H}^B &: \bigwedge \left\{\,\begin{aligned}
    \mathbf{H}^A &\ne \mathbf{H}^B \\
    \mathbf{H}^A_T &= \mathbf{H}^B_T \\
    \mathbf{H}^A &\in \mathbf{A}(\mathbf{H}^\best) \\
    \mathbf{H}^A &\not\in \mathbf{A}(\mathbf{H}^\final)
  \end{aligned}\right.
\end{align}

Of these acceptable blocks, that which contains the most ancestor blocks whose author used a seal-key ticket, rather than a fallback key should be selected as the best head, and thus the chain on which the participant should make \textsc{Grandpa} votes.

Formally, we aim to select $\mathbf{B}^\best$ to maximize the value $m$ where:
\begin{equation}
  m = \sum_{\mathbf{H}^A \in \mathbf{A}^\best} \mathbf{T}^A
\end{equation}
