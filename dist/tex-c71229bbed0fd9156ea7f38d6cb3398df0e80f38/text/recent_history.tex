\section{Recent History}\label{sec:recenthistory}

We retain in state information on the most recent $\mathsf{H}$ blocks. This is used to preclude the possibility of duplicate or out of date work-reports from being submitted.
\begin{equation}
  \beta \in \lseq\ltuple \isa{h}{\H}\ts\isa{\mathbf{b}}{\seq{\H?}}\ts\isa{s}{\H}\ts\isa{\mathbf{p}}{\lseq\H\rseq_{:\mathsf{C}}} \rtuple\rseq_{:\mathsf{H}}
\end{equation}

For each recent block, we retain its header hash, its state root, its accumulation-result \textsc{mmr} and the hash of each work-report made into it which is no more than the total number of cores, $\mathsf{C} = 341$.

During the accumulation stage, a value with the partial transition of this state is provided which contains the update for the newly-known roots of the parent block:
\begin{equation}
  \beta^\dagger \equiv \beta\quad\exc\quad\beta^\dagger[|\beta| - 1]_s = \mathbf{H}_r
\end{equation}

We define an item $n$ comprising the new block's header hash, its accumulation-result Merkle tree root and the set of work-reports made into it (for which we use the guarantees extrinsic, $\xtguarantees$). Note that the accumulation-result tree root $r$ is derived from $\beefycommitmap$ (defined in section \ref{sec:accumulation}) using the basic binary Merklization function $\mathcal{M}_B$ (defined in appendix \ref{sec:merklization}) and appending it using the \textsc{mmr} append function $\mathcal{A}$ (defined in appendix \ref{sec:mmr}) to form a Merkle mountain range.
\begin{equation}
  \begin{aligned}
    \label{eq:buildbeefymap}
    \using r &= \mathcal{M}_B(\orderby{s}{\se_4(s)\frown\se(h) \mid (s, h) \in \beefycommitmap}, \mathcal{H}_K) \\
    \using \mathbf{b} &= \mathcal{A}(\text{last}(\sq{\sq{}} \concat \sq{x_\mathbf{b} \mid x \orderedin \beta}), r, \mathcal{H}_K) \\
    \using n &= \tup{
      \is{\mathbf{p}}{[((g_w)_s)_h \mid g \orderedin \xtguarantees]}\ts
      \is{h}{\mathcal{H}(\mathbf{H})}\ts\mathbf{b}\ts\is{s}{\H^0}
      }
    \end{aligned}
  \end{equation}
  
  The state-trie root is as being the zero hash, $\H^0$ which while inaccurate at the end state of the block $\beta'$, it is nevertheless safe since $\beta'$ is not utilized except to define the next block's $\beta^\dagger$, which contains a corrected value for this.
  
  The final state transition is then:
  \begin{equation}
    \beta' \equiv {\overleftarrow{\beta^\dagger \doubleplus n}}^\mathsf{H}
  \end{equation}
  