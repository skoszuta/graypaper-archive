\section{Conclusion}\label{sec:conclusion}

We have introduced a novel computation model which is able to make use of pre-existing crypto-economic mechanisms in order to deliver major improvements in scalability without causing persistent state-fragmentation and thus sacrificing overall cohesion. We call this overall pattern collect-refine-join-accumulate. Furthermore, we have formally defined the on-chain portion of this logic, essentially the join-accumulate portion. We call this protocol the \Jam chain.

We argue that the model of \Jam provides a novel ``sweet spot'', allowing for massive amounts of computation to be done in secure, resilient consensus compared to fully-synchronous models, and yet still have strict guarantees about both timing and integration of the computation into some singleton state machine unlike persistently fragmented models.

\subsection{Further Work}

While we are able to estimate theoretical computation possible given some basic assumptions and even make broad comparisons to existing systems, practical numbers are invaluable. We believe the model warrants further empirical research in order to better understand how these theoretical limits translate into real-world performance. We feel a proper cost analysis and comparison to pre-existing protocols would also be an excellent topic for further work.

We can be reasonably confident that the design of \Jam allows it to host a service under which Polkadot \emph{parachains} could be validated, however further prototyping work is needed to understand the possible throughput which a \textsc{pvm}-powered metering system could support. We leave such a report as further work. Likewise, we have also intentionally omitted details of higher-level protocol elements including cryptocurrency, coretime sales, staking and regular smart-contract functionality.

A number of potential alterations to the protocol described here are being considered in order to make practical utilization of the protocol easier. These include:

\begin{itemize}
  \item Synchronous calls between services in accumulate.
  \item Restrictions on the \texttt{transfer} function in order to allow for substantial parallelism over accumulation.
  \item The possibility of reserving substantial additional computation capacity during accumulate under certain conditions.
  \item Introducing Merklization into the Work Package format in order to obviate the need to have the whole package downloaded in order to evaluate its authorization.
\end{itemize}

The networking protocol is also left intentionally undefined at this stage and its description must be done in a follow-up proposal.

Validator performance is not presently tracked on-chain. We do expect this to be tracked on-chain in the final revision of the \Jam protocol, but its specific format is not yet certain and it is therefore omitted at present.
