\section{Bandersnatch Ring VRF}\label{sec:bandersnatch}

The Bandersnatch curve is defined by \cite{cryptoeprint:2021/1152}.

The singly-contextualized Bandersnatch Schnorr-like signatures $\bandersig{k}{c}{m}$ are defined as a formulation under the \emph{ietf} \textsc{vrf} template specified by \cite{hosseini2024bandersnatch} (as IETF VRF) and further detailed by \cite{rfc9381}.

\begin{align}
  \bandersig{k \in \H_B}{c \in \H}{m \in \Y} \subset \Y_{96} &\equiv \{ x \mid x \in \Y_{96}, \text{verify}(k, c, m, \text{decode}(x_{\dots32}), \text{decode}(x_{32\dots})) = \top \}  \\
  \banderout{s \in \bandersig{k}{c}{m}} \in \H &\equiv \text{hashed\_output}(\text{decode}(x_{\dots32}) \mid x \in \bandersig{k}{c}{m})
\end{align}

The singly-contextualized Bandersnatch Ring\textsc{vrf} proofs $\bandersnatch{r}{c}{m}$ are a zk-\textsc{snark}-enabled analogue utilizing the Pedersen \textsc{vrf}, also defined by \cite{hosseini2024bandersnatch} and further detailed by \cite{cryptoeprint:2023/002}.

\begin{align}
  \mathcal{O}(\seq{\H_B}) \in \Y_R &\equiv \text{KZG\_commitment}(\seq{\H_B})  \\
  \bandersnatch{r \in \Y_R}{c \in \H}{m \in \Y} \subset \Y_{784} &\equiv \{ x \mid x \in \Y_{784}, \text{verify}(r, c, m, \text{decode}(x_{\dots32}), \text{decode}(x_{32\dots})) = \top \}  \\
  \banderout{p \in \bandersnatch{r}{c}{m}} \in \H &\equiv \text{hashed\_output}(\text{decode}(x_{\dots32}) \mid x \in \bandersnatch{r}{c}{m})
\end{align}

Note that in the case a key $\H_B$ has no corresponding Bandersnatch point when constructing the ring, then the Bandersnatch \emph{padding point} as stated by \cite{hosseini2024bandersnatch} should be substituted.